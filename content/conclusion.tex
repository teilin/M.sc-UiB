Målet med dette arbeidet har vært å å se på utfordringer innen vedlikeholdsplanlegging i oljeindustrien. Det går ut på å prøve ut forskjellige løsningsstrategier for å ta hånd om komplekse problemer, i et veldig dynamisk miljø som olje og gass industrien.

Den første løsningsstrategien (LS1) som tildelte kraner først løser alle probleminstansene, uavhengig av størrelsen og kompleksiteten. Løsningene ligger her 74.5\% og 81.0\% over teoretisk nedregrense, for implementasjonen uten varmebegrensing og deretter med varmebegrensning.

Den andre løsningsstrategien (LS2) som tildeler starttidspunkt til alle aktivitene først, løser ikke alle probleminstansene, men har de beste løsningene. Her er det også implementasjonen uten varmebegrensing som har de beste løsningene.

Løsningene med varmebegrensning var med begge løsningsstrategiene noe dårligere enn uten varmebegrensning. Det ser ut som ut ifra disse resultatene det er mer på løsningsstrategien hvor mange probleminstanser som blir løst. Det ser ut som det å legge til varmebegrensing ikke forbedret makespan og det må tas besluttninger på om det er best mulig løsninger eller flest mulig løsninger som er ønskelig når det skal utvikles en appliksjon med Scheduler.