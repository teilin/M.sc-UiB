Målet med denne masteroppgaven var å se på utfordringer innen automatisk vedlikeholdsplanlegging i oljeindustrien. Det er prøvd ut forskjellige løsningsstrategier samt lagt til en varmebegrensning. Alt for å ta hånd om vedlikeholdsplanlegging i olje og gassindustrien som har høye krav høye krav til planlegging, kvalitetssikring og gjennomføring.

Den første løsningsstrategien (LS1), som tildelte kraner først, løste alle probleminstansene uavhengig av størrelse og kompleksitet. For implementasjonen uten varmebegrensing og deretter med varmebegrensning ligger løsningene henholdsvis 74.5\% og 81.0\% over teoretisk nedregrense. 

Den andre løsningsstrategien (LS2), som tildeler starttidspunkt til alle aktivitene først, løste ikke alle probleminstansene, men har de beste løsningene i forhold til teoretisk nedregrense. For implementasjonen uten varmebegrensning og deretter med varmebegrensning ligger løsningene henholdsvis 14.2\% og 23.2\% over teoretisk nedregrense.

Det kan være hensiksmessig å vurdere hvilken løsningsstrategi som skal brukes avhengig av om det er ønskelig med å ha høy løsningsgrad eller bedre makespan.

Løsningene med varmebegrensning ga med begge løsningsstrategiene noe høyere makespan enn uten varmebegrensning. Dette kan være både en reell økning av nedre grense eller være en økning av at problemet er mer komplekst å løse, eller en kombinasjon av begge. Resultatene viser at løsningsstrategien har større betydning enn varmebegrensningen når det gjelder hvor mange probleminstanser som blir løst. Det ser ut som det å legge til varmebegrensing ikke forbedrer makespan og det må tas beslutninger på om det er best mulig løsninger eller flest mulig løsninger som er ønskelig når det skal utvikles en appliksjon med Scheduler.