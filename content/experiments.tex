Eksperimenteringen er utført på en Mackbook Air med 1.8 GHz Intel Core i7 prosessor og 4 GB 1333 MHz DDR3 minne. Det er totalt sett 65 benchmarksett som implementasjonene er evaluert på, hvor mange det ble funnet en løsning varierte med eller uten tilleggsressursene og tidsgrensen som var satt.

\begin{table}[h]
\begin{center}
\begin{tabular}{ | c | c | c | c | c | c | c | c | c | c | c | c |}
\hline
\textbf{Modell} & \multicolumn{6}{|c|}{\textbf{2 kraner}} & \multicolumn{6}{|c|}{\textbf{3 kraner}} & \multicolumn{2}{|c|}{\textbf{Alle}} \\ \hline
$\sharp Act(\sharp P)$ & \multicolumn{2}{|c|}{$< 100 (10)$} & \multicolumn{2}{|c|}{$> 100 \text{ og } < 1000 (45)$} & \multicolumn{2}{|c|}{$> 1000 (10)$} & \multicolumn{2}{|c|}{$< 100 (10)$} & \multicolumn{2}{|c|}{$> 100 \text{ og } < 1000 (45)$} & \multicolumn{2}{|c|}{$> 1000 (10)$} & \multicolumn{2}{|c|}{(65)} \\ 
\hline
Modell & $w_{rq}$ & $\%^{(1)}$& $w_{rq}$ & $\%^{(1)}$ & $w_{rq}$ & $\%^{(1)}$ & $w_{rq}$ & $\%^{(1)}$  & $w_{rq}$ & $\%^{(1)}$ & $w_{rq}$ & $\%^{(1)}$ & $w_{rq}$ & $\%^{(1)}$ \\ \hline
LS1 Uten varme & 1.223 & 80 & 1.153 & 44 & 1.100 & 20 & 0 & 0 & 0 & 0.031 & 49 \\
\hline
\end{tabular}
\end{center}
\caption{Relativ optimalitets index $w_{rq}$ for de forskjellige modeller}
\label{tab:resultaterSum}
\end{table}
$^{(1)}$ prosentandel løste probleminstanser

En måling av relativ kvalitet er brukt for å evaluere resultatene fra forskjellige strategier. Den avledede variabelen $w_{rq}$ er gitt ved (\ref{eq:relativkvalitet}).
\begin{equation}
w_{rq} = \frac{1}{| P_{sol} |} \sum_{P \in P_{sol}} \frac{w_{ms}(P)}{c_{lb,ms}(P)}
\label{eq:relativkvalitet}
\end{equation}
$P_{sol}$ er det settet med probleminstanser som er løst ved hver enkelt løsningsstrategi. Verdiene av $P_{sol}$ varierer fra løsningsstrategi til løsningsstrategi og disse gjennomsnittene er derfor ikke helt sammenlignbare, men de vil gi en indikasjon på kvaliteten på løsningen. $w_{rq}$ skiller ikke på strategier som feiler med å finne løsninger, men hvor robust løsningen er vil antallet løste probleminstanser indikere. Resultatene er summert opp i tabell \ref{tab:resultaterSum}.

\subsection{Uten tilleggsressurser}


\subsection{Med tilleggsressurser}
