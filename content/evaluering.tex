I denne delen blir probleminstanser, løsningsstrategier og løsninger evaluert. Denne delen blir avsluttet med å evaluere løsningene funnet i dette prosjektet opp mot løsningene funnet i artikkelen til Bård Henning Tvedt.

\subsection{Probleminstanser}
Probleminstansene har fra 50-5000 aktiviteter, mens antall lokasjoner og mannskaper i de genererte probleminstansene er henholdsvis 10 og 5.

Det ble først eksperimentert litt med mannskapets varme og lokasjoners varmekapsitet og i det første settet med probleminstanser var det overlapp mellom mannskapers varme og en lokasjons varmekapasitet. Dette førte til veldig få løsninger når varmebegrensningen ble tatt med. Det ble funnet løsninger på noen av probleminstansene under 100 aktiviteter, mens over 100 aktiviteter var det kun en løsning. Det er grunn til å tro dette var tilfeldig, ut i fra at varme og varmekapasitet blir tilfeldig plassert innenfor et gitt område. Erfaringene fra dette er kun tatt til etterretning og resultatene er ikke interessante her pga. få løsninger.

Probleminstansene hvor varme og varmekapasitet ikke er overlappende, ga flere løsninger og er brukt videre gjennom prosjektoppgaven.

\subsection{Løsningsstrategier}
Tabell \ref{tab:resultaterSum100s} viser et ganske klart bilde av de sterke og svake sidene til de to løsningsstrategiene som er brukt. LS1 løser alle probleminstanser, mens LS2 gir betydelig bedre makespan.

Ser vi samtidig på ressursskjemaene for $LS1\sharp1\sharp3$, figur \ref{fig:RessursWithoutAct50Loc10Crew5Crane2LS1}, og $LS2\sharp1\sharp3$, figur \ref{fig:ResourceAct50Crane2LS2_UtenVarme_100s}, for gitte tilfeller, ser vi følgende:
\begin{itemize}
\item $LS1\sharp1\sharp3$ har få aktiviteter mot enden av tidsaksen. Det er langere makespan.
\item $LS2\sharp1\sharp3$ har jevnt med aktiviteter langs tidsaksen. Det er kortere makespan.
\end{itemize}
Dette har mest sannsynlig sin årsak i at LS1 produserer flere løsninger fordi begrensningene er på plass tidlig i søkeprosessen.
LS2 produserer bedre løsninger fordi det tildeles startverdier til aktiviteter uten å betrakte kranfordelingen så nøye. Men antallet løsninger blir gjerne redusert fordi søkeprosessen ikke klarer å komme ut av konfliktene som oppstår når man tildeler kran på slutten.

\subsection{Resultater}
I denne prosjektoppgaven er en eksisterende løsning utvidet med en ressurs kalt varmeressurs. For å evaluere resultatene fra den utvidete løsningen, evalueres den utvidete løsningen med 10 lokasjoner opp mot den opprinnelige løsningen med 25 lokasjoner. Hensikten med å sammenligne resultatene fra den opprinnelige løsningen med 25 lokasjoner og 10 lokasjoner er å analysere effekten av antall lokasjoner.

Evalueringen innebærer også å analysere effekten av varmebegrensningen, som er gjort ved å sammenligne resultatene med 10 lokasjoner med og uten varmebegrensning. Evalueringen av resultatene baserer seg på gjennomsnittsverdiene i tabell \ref{tab:resultaterSum100s}.

\subsubsection{10 Lokasjoner}
Løsningene er i ganske god overensstemmelse med forventningene i kapitel \ref{sec:implprocess}. Med LS1 hvor kranfordelingen ble valgt først var gjennomsnittlig makespan dårligere enn med LS2.

\begin{figure}[!h]
\centering
\includegraphics[scale=0.3]{content/gfx/Act50Crane2AssignHeatResourceCrane}
\caption{Ressursoversikt på Act50Loc10Crew5Crane2 med varmebegrensning, LS1, $\sharp2\sharp3$}
\label{fig:RessursWithAct50Loc10Crew5Crane2LS1}
\end{figure}
\begin{figure}[!h]
\centering
\includegraphics[scale=0.3]{content/gfx/Act50Crane2WithoutResourceCrane}
\caption{Ressursoversikt på Act50Loc10Crew5Crane2 uten varmebegrensning, LS1, $\sharp1\sharp3$}
\label{fig:RessursWithoutAct50Loc10Crew5Crane2LS1}
\end{figure}
\begin{figure}[!h]
\centering
\includegraphics[scale=0.3]{content/gfx/Act50Crane3AssignHeatResourceCrew}
\caption{Ressursoversikt på Act50Loc10Crew5Crane3 med varmebegrensning, LS1 og $\sharp2\sharp3$}
\label{fig:RessursWithAct50Loc10Crew5Crane3LS1}
\end{figure}
Figur \ref{fig:RessursWithoutAct50Loc10Crew5Crane2LS1} er en gitt situasjon uten varmebegrensning og figur \ref{fig:RessursWithAct50Loc10Crew5Crane2LS1} er den samme med varmebegrensning. Begge figurene har lokasjon 2 fullt utnyttet. Kran 1 er på lokasjonen og er den reelle begrensningen. Innføring av varmebegrensningen gir i dette tilfellet med kran som begrensning lite utslag i henhold til tabell \ref{tab:resultaterSum100s} (liten økning i makespan). Generelt er det forventet en økning i makespan når varmebegrensing innføres. Denne økningen antas ikke i sin helhet å kun komme fra varmebegrensningen, men sannsynligvis også være et resultat at det blir vanskeligere å løse problemet.

Figur \ref{fig:RessursWithAct50Loc10Crew5Crane3LS1} kontra figur \ref{fig:RessursWithoutAct50Loc10Crew5Crane2LS1} er en gitt situasjon og økning fra til tre kraner. Det er samme betrakning rundt utnyttelse, men i figur \ref{fig:RessursWithAct50Loc10Crew5Crane3LS1} endres dette. Her er det mannskap som over tid bruker hele kapasiteten.

\begin{equation}
c_{load}(Crane_{k}) = \sum_{c_{Crane}(Act_{i})=Crane_{k}} c_{dur}(Act_{i})
\label{eq:kranstyrke}
\end{equation}
Ved å sammenligne om verdier fra (\ref{eq:kranstyrke}) er mindre eller større enn (\ref{eq:mannskapsstyrke}), gir det et uttryk for hvor sterk eller svak kranressursen er. Hvis (\ref{eq:kranstyrke}) er mindre vil det si at kranressursen er svak. Ut ifra dette er det en signifikant forskjell på to og tre kraner. Ved to kraner er kranressursen veldig sterke, mens kranressursen ikke er fullt så sterke med tre kraner.

Tiden det tar å løse probleminstansene er også veldig lik uten noe spesielt mønster både med og uten varmebegrensning. Det er derfor litt bedre løsninger uten varmebegrensning.

Ved å ta utgangspunkt i LS2 uten varmebegrensing og uten sikkerhetsbegrensing på kran fra tabell \ref{tab:resultaterSum100s} er makespan totalt sett 1.0\% over teoretisk nedregrense med tidsgrense på 100 sekunder. Det er også løsninger på alle probleminstanser. Denne løsningen er veldig lite begrenset, da det ikke er noe grense for hvor mange mannskaper som kan være på en lokasjon og heller ingen begrensing på gjennomføring av aktiviteter ved kranbruk. Når sikkerhetsbegrensingene blir lagt til øker makespan med 13.2\% og det er ikke løsninger på mer enn 36.88\% av probleminstansene. Med tanke på at sikkerhetsbegrensingene vil opprette sikkerhetssoner som vil stenge lokasjonen til kranen og lokasjonen aktiviteten som bruker kranen blir utført på. Yterligere øker makespan med 9\% og det blir funnet løsninger på 40\% av probleminstansene da varmebegrensingen blir lagt til. Når varmebegrensigen blir lagt til, begrenses antall mannskaper som kan jobbe på lokasjonene samtidig. Det kan se ut som Scheduler sliter med sikkerhetsbegrensningene på kran.

\subsubsection{10 lokasjoner mot 25 lokasjoner}
For å sammenligne løsningene i denne prosjektoppgaven med 10 lokasjoner, mot løsningene til \bht med 25 lokasjoner, er tabell \ref{tab:resultaterSumTvedt} hentet fra \cite{tvedtbezem}. Det benyttes henholdsvis $LS2\sharp1\sharp3$ og Inferred.
\begin{table}[!h]
\caption{Relativ optimalitets indeks $w_{rq}$ for de forskjellige modellene}
\begin{center}
\begin{tabular}{ | c | c | c | c | c | c | c | c | c | c | c | }
\hline
\textbf{Modell} & \multicolumn{4}{|c|}{\textbf{2 kraner}} & \multicolumn{4}{|c|}{\textbf{3 kraner}} & \multicolumn{2}{|c|}{\textbf{Alle}} \\ \hline
$\sharp Act(\sharp P)$ & \multicolumn{2}{|c|}{$< 100 (25)$} & \multicolumn{2}{|c|}{$> 100 (23)$} & \multicolumn{2}{|c|}{$< 100 (28)$} & \multicolumn{2}{|c|}{$> 100 (23)$} & \multicolumn{2}{|c|}{(99)} \\ 
\hline
Modell & $w_{rq}$ & $\%^{(2)}$ & $w_{rq}$ & $\%^{(2)}$  & $w_{rq}$ & $\%^{(2)}$ & $w_{rq}$ & $\%^{(2)}$ & $w_{rq}$ & $\%^{(2)}$ \\ \hline
Greedy & 1.311 & 100 & 1.524 & 100 & 1.276 & 100 & 1.393 & 100 & 1.370 & 100 \\
Inferred & 1.068 & 100 & 1.063 & 35 & 1.017 & 32 & 1.000 & 4 & 1.055 & 43 \\
Under$^{(1)}$ & 1.032 & 100 & 1.014 & 100 & 1.015 & 100 & 1.002 & 96 & 1.016 & 99 \\
Over$\sharp 1$ & 1.143 & 100 & 1.076 & 100 & 1.176 & 100 & 1.037 & 91 & 1.114 & 98 \\
Over$\sharp 2$ & 1.104 & 100 & 1.047 & 96 & 1.068 & 96 & 1.005 & 65 & 1.063 & 90 \\
\hline
\multicolumn{11}{l}{\begin{minipage}{6in}$^{(1)}$ Denne modellen garanterer ikke gyldige løsninger.
$^{(2)}$ prosentandel løste instanser \end{minipage}}
\end{tabular}
\end{center}
\label{tab:resultaterSumTvedt}
\end{table}
Det er stort samsvar mellom både makespan og løsningsgrad. Makespan er bra og varierer lite. Løsningsgrad reduseres svært mye med mange aktiviteter, både for to og tre kraner. I begge modellene tildeles tid til aktiviteter før kranressurser. Som tidligere også beskrevet om LS2 antas dette å skyldes at kranressurser skal tildeles i et økende antall allerede disponerte aktiviteter.