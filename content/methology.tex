I denne delen, blir verktøy og teknologier som er brukt i prosjektet beskrevet. I tilegg vil forskingsmetoder som er brukt bli beskrevet og beskrivelse av strategiene for evaluering av løsningene.

\subsection{Verktøy brukt i prosjektet}
De følgende verktøyene og teknologiene utviklet av IBM var brukt for å gjennomføre formålet med prosjektet.

\subsection{Kort om IBM ILOG Concert Technology}\nomenclature{Concert}{IBM ILOG Concert Technology}
Concert er et C++ bibliotek med funksjoner som gir mulighet til å designe modeller av problemer innen matematisk programmering og innen begrensningsprogrammering. Det er ikke noe eget programmeringsspråk, som da gir muligheter til å bruke datastrukturer og kontrollstrukturer som allerede finnes i C++. Igjen så gir det gode muligheter til å integrere Concert i allerede eksisterende løsninger og systemer. Alle navn på typer, klasser og funksjoner har prefiksen Ilo.

De enkleste klassene (eks. IloNumVar og IloConstraint) i Concert har også tilhørende en klasse med matriser hvor matrisen er instanser av den enkle klassen. Et eksempel på det er IloConstraintArray er instanser av klassen IloConstraint.\cite{cpconcertilog}

Concert gjør det mulig å lage en modell av optimaliseringsproblemer uavhengig av algoritmene som er brukt for å løse det. Det tilbyr en utvidelse  modelerings lag tatt fra flere forskjellige algoritmer som er klare til å brukes ut av boksen. Dette modeleringslaget gjør det mulig å endre modellen uten å skrive om applikasjonen.\cite{cpsolverilog}

\subsection{Kort om IBM ILOG Solver}\nomenclature{Solver}{IBM ILOG Solver}
IBM ILOG Solver er et C++ bibliotek utviklet for å løse komplekse kombinatoriske problemer innen forskjellige områder. Eksempler på anvendelsesområder kan være produksjonsplanlegging, resurs tildeling, timeplanplanlegging, personellplanlegging, osv. Solver er basert på Concert. Som i Concert, så er heller ikke Solver noe eget programmeringsspråk, som gir mulighetene til å bruke egenskapene til C++.

Det å gjøre det enkelest mulig å omgjøre applikasjoner fra plattformer til plattformer, Solver og Concert utelukkes karaktertrekk som skiller seg fra forskjellige systemer. Av den grunn, anbefales det å bytte ut de enkle typene i C++ med ILOG sine egne:
\begin{itemize}
\item IloInt som er signed long integers
\item IloAny som er pekere
\item IloNum som er double presisjon floating-point verdier
\item IloBool som er boolean verdier: IloTrue og IloFalse
\end{itemize}
Solver bruker begrensningsprogrammering for å finne løsninger til optimaliseringsproblemer. Det å finne løsninger med Solver er basert på tre steg: beskrive, modell og løse. De tre stegene nærmere forklart følger:

Først må problemet beskrives i programmeringsspråket som brukes.

Det andre steget er å bruke Concert klassene for å opprette en modell av problemet. Modellen blir da satt sammen av besluttningsvariable og begrensninger. Besluttningsvariablene er den ukjente informasjonen i problemet som skal løses. Alle besluttingsvariablene har et domene med mulige verdier. Begrensningene setter grensene for kombinasjonene av verdier for de besluttingsvariablene.

Det siste steget er å bruke Solver for å løse problemet. Det inneholder å finne verdier for alle besluttingsvariablene samt ikke bryte noen av de definerte begrensningene og dermed enten maksimere eller minimere målet, hvis det er et mål inkludert i modellen. Solver ser etter løsninger i et søkeområdet. Søkeområdet er alle mulige kombinasjoners av verdier.\cite{cpsolverilog}

\subsection{Kort om IBM ILOG Scheduler}\nomenclature{Scheduler}{IBM ILOG Scheduler}
IBM ILOG Scheduler hjelper med å utvikle problemløsnings-applikasjoner som krever behandling av ressurser fordelt på tid. Scheduler er et C++ bibliotek som baserer seg på Solver, og som Solver, så gir det alle mulighetene med objektorientering og begrensningsprogrammering. Scheduler har spesifisert funksjonalitet på å løse problemer innen planlegging og ressurs tildeling.\cite{cpschedulerilog}

\subsection{Forskningsmetoder}
Forskningsmetoden som er brukt i prosjektet er å eksperimentere med implementasjonen av ressursene og løsningsstrategien.

\subsubsection{Implementeringsprosessen}
Problemet med å finne ut om en RCPSP løsning i tillegg til sikkerhetsbegrensningene med makespan mindre enn en gitt frist finnes er NP-hard. Dette betyr at den utvidede problemet med varme ressursen også må være NP-hard og derfor en optimal løsning med til og med de enkleste formene av problemet er ikke garantert innen polynomisk tid. To forskjellige løsningsstrategier er testet. Begge løsningsstrategiene er implementert i IBM ILOG Solver og IBM ILOG Scheduler biblotekene. I begge løsningsstrategiene brukes standard søkemål i ILOG Solver.

Den første løsningsstrategien bruker IloAssignAlternatives, som blir brukt til å tildele kraner til aktivitetene. Den neste søkemålet er IloRankForward... og tilslutt brukes søkemålet IloSetTimesForward ...

Den andre
%IloSetTimesForward & IloAssignAlternatived & IloRankForward

\colorbox{red}{Denne delen skal utvides ytterligere.}

\subsubsection{Evaluering av prosessen}
Prosessen med eksperimenteringen av implementasjonen blir evaluert ved å undersøke makespan opp mot teoretisk øvregrense og teoretisk nedregrense i både løsningene uten tilleggsressurser og med tilleggsressurser. Løsningene fra prosessen med og uten tilleggsresursene vil også bli evaluert opp mot hverandre. Dette innebærer bruk av kvantitative metoder.

Forskningsmetoden vil bli evaluert ved å bruke genererte benchmarksett, som er generert av et eksternt program. Benchmarksettene som genereres kan bestemmes hvor mange av de forskjellige ressursene som skal være med i benchmarksettet. I dette prosjektet er det et sprang på 50 - 5000 aktiviteter som implementasjon blir evaluert på.

\subsection{Evalueringsstrategi}
For å evaluere kvaliteten på løsningene, er teoretisk- øvregrense og nedregrense for makespan blir kalkulert.

\subsubsection{Teoretisk nedregrense}
En teoretisk nedregrense er kalkulert basert på ressurstilgjengeligheten for den mest begrensede mannskapet. Resultatløsningen blir kanskje ikke gyldig for hele problemet, men er komprimert tett sammen for mannskapet og utnytte hver eneste mannskapstime.
\begin{equation}
c_{load}(Crew_{j}) = \sum_{c_{crew}(Act_{i}) = Crew_{j}} c_{dur}(Act_{i})
\end{equation}
\begin{equation}
c_{reload}(Crew_{j}) = \frac{c_{load}(Crew_{j})}{c_{cap}(Crew_{j})}
\end{equation}
\begin{equation}
c_{lb,ms}(P) = max_{j}\{ c_{reload}(Crew_{j}) \}
\end{equation}

\subsubsection{Teoretisk øvregrense}
Teoretisk øvregrense for makespan er
\begin{equation}
c_{ub,ms}(P) = c_{hor}(P) = \sum_{i} c_{dur}(Act_{i})
\end{equation}
som indikerer at i det verste tilfelle blir alle aktivitetene utført etter hverandre, en om gangen.