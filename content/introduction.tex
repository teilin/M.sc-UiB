Denne oppgaven er en utvidelse av \bht sitt arbeid med "Automatisk Planlegging i Oljeindustrien (APO)"\nomenclature{APO}{Automatisk Planlegging i Oljeindustrien} \cite{tvedtbezem}. APO er en planleggingsmetodikk for å løse et problem med planlegging av vedlikeholdsaktiviter i oljeindustrien. Det blir brukt en fiktiv oljeplatform for å kunne gjøre problemet så likt som mulig virkeligheten. Implementasjonen i IBM ILOG Scheduler (Scheduler) er utvidet med ressurser på det som kalles varmebegrensning. Med bruk av varme er det ikke å forstå som varme i tradisjonell forstand, men et mål for hvor mye belastning en lokasjon på platformen tåler. Løsningene blir evaluert med et eksternt program, som kalkulerer teoretisk- øvregrense og nedregrense og sjekker om begrensningene er tilfredstilt.

\subsection{Beskrivelse av kommende kapitler}
I kapittelet om metode, vil fremgangsmåten for prosjektet bli lagt frem og hvordan løsningene har blitt evaluert. Under kapittelet om eksperimenteringen vil det legges frem løsninger med forskjellige strategier, både med og uten varmebegrensning. Løsningene vil også bli evaluert og til slutt vil det bli sammenfattet en konklusjon over det arbeidet som er gjort.