Denne oppgaven er en utvidelse av \bht sitt arbeid ''Automatisk Planlegging i Oljeindustrien (APO)"\nomenclature{APO}{Automatisk Planlegging i Oljeindustrien} \cite{tvedtbezem}. APO er en planleggingsmetodikk for planlegging av vedlikeholdsaktiviter i oljeindustrien. Det blir brukt en fiktiv oljeplatform for å kunne gjøre problemet så likt som mulig virkeligheten. Implementasjonen i IBM ILOG Scheduler (Scheduler) er utvidet med en begrensning som kalles varme. Varme i denne sammenheng er ikke å forstå som varme i tradisjonell forstand, men et mål for hvor mye belastning en lokasjon på platformen tåler. Løsningene blir evaluert med et eget (eksternt) program, som kalkulerer teoretisk- øvregrense og nedregrense og sjekker om begrensningene er tilfredstilt.

\subsection{Beskrivelse av kommende kapitler}
I kapittelet (kapitlet?) om metode vil både fremgangsmåten for prosjektet bli lagt frem og hvordan løsningene har blitt evaluert. Under kapittelet (?) om resultater vil det legges frem løsninger med forskjellige strategier, både med og uten varmebegrensning. Løsningene evalueres og til slutt sammenfattes en konklusjon over det arbeidet som er gjort.