Denne oppgaven er en utvidelse av \bht sitt arbeid ''Automatisk Planlegging i Oljeindustrien (APO)"\nomenclature{APO}{Automatisk Planlegging i Oljeindustrien} \cite{tvedtbezem}. APO er en planleggingsmetodikk for planlegging av vedlikeholdsaktiviter i oljeindustrien. Det blir brukt en fiktiv oljeplattform for å kunne gjøre problemet så likt som mulig virkeligheten. Vedlikeholdsaktiviteter som skal planlegges/løses genereres i et eget program. Disse, såkalte probleminstanser, blir inputdata i den IBM ILOG Scheduler (Scheduler) implementerte løsningen. Implementasjonen i Scheduler er utvidet med en begrensning som kalles varme. Varme i denne sammenheng er ikke å forstå som varme i tradisjonell forstand, men et mål for hvor mye belastning en lokasjon på plattformen tåler. Løsningene blir evaluert med et eget (eksternt) program, som kalkulerer teoretisk øvregrense og nedregrense og sjekker om begrensningene er tilfredstilt.

\subsection{Beskrivelse av kommende kapitler}
I kapitlet om metode vil det beskrives hvordan prosjektoppgaven er utført, hvilke metoder som er brukt og hvordan resultatene har blitt evaluert. Under kapitlet om resultater vil det legges frem resultater med forskjellige løsningsstrategier, med kombinasjoner av begrensninger. Løsningene evalueres og til slutt sammenfattes en konklusjon over det arbeidet som er gjort.