Denne oppgaven er en utvidelse av \bht sitt arbeid med "Automatisk Planlegging i Oljeindustrien (APO)"\nomenclature{APO}{Automatisk Planlegging i Oljeindustrien} \cite{tvedtbezem}. APO er et planleggingsverktøy for å løse et problem med planlegging av bruk av ressurser i oljeindustrien. Det blir brukt en fiktiv oljeplatform for å kunne gjøre problemet så likt som mulig virkeligheten. Implementasjonen i IBM ILOG Scheduler er utvidet med ressurser på det som kalles varmebegrensning. Med bruk av varme er det ikke å forstå som varme i tradisjonell forstand, men et mål for hvor mye kapasitet en ressurs bruker. Løsningene blir evaluert med et eksternt program.

\subsection{Beskrivelse av kommende kapitler}
I kapittelet om metode, vil fremgangsmåten for prosjektet bli lagt frem og hvordan løsningene har blitt evaluert. Under kapittelet om eksperimenteringen vil det legges frem løsninger ved forskjellige strategier og med og uten ressursene. Løsningene vil også bli evaluert og til slutt vil det bli sammenfattet en konklusjon over det arbeidet som er gjort.