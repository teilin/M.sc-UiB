\subsection{Bakgrunn}
\colorbox{red}{Bakgrunn på utvides yterligere!}
I dette prosjektet, verktøyene som er brukt for å utføre eksperimenter på emnet er ILOG Scheduler som er endel av IBM sitt ILOG CP\nomenclature{CP}{Begrensningsprogrammering (engelsk: Constraint programming)}. ILOG Scheduler er et C++ bibliotek som gjør det mulig å definere planleggingsbegrensninger i form av ressurser og aktiviteter. Planlegging er en prosess ved å tildele ressurser til aktiviteter og tildele en tid til aktiviteter så det ikke er noen konflikt med begrensingene.\cite{Pape94implementationof} Automatisk planlegging er endel av det som kalles kunstig intelligens (AI) \nomenclature{AI}{Kunstig intelligens (engelsk: artificial inteligence)}.

Problemer i kategorien \textit{beregningsvitenskap kompleksitets teori}\cite{compcomplextheory} og som også er minst like vanskelige som de vanskeligste probleme i NP-kategorien, kalles \textit{ikke-deterministisk polynomtid problemer vansklige} (NP-hard) \nomenclature{NP-hard}{Ikke-deterministisk polynomtid hard (engelsk: non-deterministic polynomial-time hard)}. Det er forskjellige kategorier av NP-harde problemer, avgjøreksesproblemer, søkeproblemer og optimaliseringsproblemer. \cite{nphardwikipedia} Problemet i dette prosjektet er i kategorien optimaliseringsproblem. Fra wikipedia \cite{optimizationproblemwiki} er et optimaliseringsproblem: "Et optimaliseringsproblem er et problem med å finne den beste løsningen ut ifra alle gyldige løsninger".

\subsubsection{Relatert arbeid}
\colorbox{red}{Skrive om tidligere gjort forskning på området.}
Nuijten og Pape har brukt ILOG Scheduler til å løse "Job Shop Scheduling" problemet \nomenclature{JSSP}{Job Shop Scheduling Problem} og det er et problem som er NP-hard. Når problemer blir løst med Scheduler blir alle starttidene til aktivitetene, i en mulig løsning, som ikke er bevist om er en gyldig løsning tatt vare på i søkefasen. Begrensingsbasert planlegging blir ofte brukt for å redusere beregningstiden som trengs. Begrensingene reduserer domenet av variable. I industrien er ofte planleggingsproblemmer uført i dynamiske miljøer, som vil øke behovet for reaktive planleggingsalgoritmer. En mulighet for å bruke begrensningsprogrammering på en reaktiv måte er å overføre handlinger og deler av planleggingen som den er nå til begrensinger og domener. \cite{Nuijten:1998:CJS:594934.594971}
\begin{itemize}
\item (Taillard) Benchmark for basic scheduling problems \cite{Taillard1993278}
\begin{itemize}
\item Vanskelige problemer; beste makespan langt unna nedre grense
\item Går på hvordan oppretter gode og vanskelige benchmarksett for 3 kjente CP problemer.
\end{itemize}
\item (Wallace) On bencgmarking CLP Platforms \cite{Wallace:2004:BCL:956860.956861}
\begin{itemize}
\item Tar for seg positive og negative aspekter ved benchmarking CLP problemer.
\item Benchmarking oppgave 2 oppgaver; 1 - dekke et representativt problem og bredt programmerings begrep
\item 2 måter; application benchmarking og unit tests
\item Utfordring opprette et klar målsetting med ytelsen.
\item Teoretisk idelle måten å velge beste CLP systemet for en applikasjon er å implementere en løsning for applikajsonen for hvert system og velge den beste. Praktisk: kan bare benchmarke systemer med predefinerte problemer og håpe at resultatet kan bli overført til applikasjonen som krever det.
\item Sammenlignet CLP problemer SICStus og IF/Prolog med andre CLP systemer (clp(FD), CHR, ILOG, Oz og B-prolog)
\item Tidsmekanismer vanskelig spesielt i høynivåspråk, hvor det ofte måles CPU tid.
\item Sammenligne CLP systemer gir flere tilfeller som ikke har oppstått ved sammenliggning av mer tradisjonelle systemer. Utførelsestiden for 2 CLP språk løser samme problem ofte avhenger mindre på språkimplementasjon enn presis algoritme utførelse av programmet.
\item For å benchmarke forskjellige CLP språk støtter samme funskjonalitet er det nødt å skrive program i forskjellig språk med samme algoritme.
\end{itemize}

\item (Laborie) IBM ILOG CP Optimizer for Detailed Scheduling Illustrated in Three problems \cite{Laborie:2009:IIC:1560579.1560593}
\begin{itemize}
\item ...
\end{itemize}

\item (Biskup) Enslig-maskin planlegging med læringsbetraktninger %\cite{Biskup1999173}
\begin{itemize}
\item \dots
\end{itemize}

\end{itemize}

\subsection{Målet med prosjektet}
Målet med prosjektet er todelt, og består i å vurdere den modifiserte problemstillingen mot den opprinnelige i forhold til:
\begin{itemize}
\item minimere \textit{makespan}
\item antall begrensninger
\item implementasjon i \ilog
\end{itemize}

I den opprinnelige problemstillingen vil noen aktiviteter være relativt lite begrenset. Dette gjør at løsningsrommet er stort, og traverseringen opp og ned i søketreet tar lang tid. På tross av et antatt stort løsningsrom så sliter den \ilog implementerte løsningsstrategien med å finne løsninger i mange av probleminstansene.

\begin{itemize}
\item Vil flere begrensninger gjøre det lettere å finne en løsning?
\item Er det noe spesielt med akkurat disse instansene eller er det implementasjon i \ilog som er årsaken?
\end{itemize}