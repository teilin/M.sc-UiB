\subsection{Bakgrunn}
Begrensningsprogrammering er en programmeringsparadigme hvor relasjoner mellom variable blir satt i form av begrensninger. Begrensninger er en form for deklarativ programmering, som skiller seg fra den mer vanlige imperativ programmeringsspråk\footnote{Imperativ progrgrammeringspråk har sekvenser med som blir utført.} ved at løsningen blir til ved å tilfredstille begrensningene. Det er forskjellige områder i begrensningsprogrammering som "Consreaint Satisfaction problems" og planleggingsproblemer. Det mest kjente planleggingsproblemet er "Job Shop Scheduling".\cite{cpwikipedia}

I dette prosjektet, værktøyene som er brukt for å utføre eksperimenter på emnet er ILOG Schuduler som er endel av IBM sitt ILOG CP\nomenclature{CP}{Constraint programming}. ILOG Scheduler er et C++ biblotek som gjør det mulig å definere planleggingsbegrensninger i form av ressurser og aktiviteter. Planlegging er en prosess ved å tildele ressurser til aktiviteter og tildele en tid til aktivitene så det ikke er noen konflikt med begrensingene.\cite{Pape94implementationof}

Det overordnede målet med denne avhandlingen er å utvide ILOG Scheduler løsningen til \bht med flere resursser for å sjekke om det å legge til flere ressurser vil gi bedre og flere løsninger enn de opprinnelige løsningene til \bht.

\subsection{Målet med prosjektet}
Målet med prosjektet er todelt, og består i å vurdere den modifiserte problemstillingen mot den opprinnelige i forhold til:
\begin{itemize}
\item antall begrensninger
\item implementasjon i \ilog
\end{itemize}

I den opprinnelige problemstillingen vil noen aktiviteter være relativt lite begrenset. Dette gjør at løsniingsrommet er stort, og traverseringen opp og ned i søketreet tar lang tid.

På tross av et antatt stort løsningsrom så sliter den \ilog implementerte løsningsstrategien med å finne løsninger i mange av probleminstansene.

\begin{itemize}
\item Vil flere begrensninger gjøre det lettere å finne en løsning?
\item Er det noe spesielt med akkurat disse instansene eller er det implementasjonen i \ilog som er årsaken?
\end{itemize}