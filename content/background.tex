\subsection{Bakgrunn}
I dette prosjektet, verktøyene som er brukt for å utføre eksperimenter på emnet er ILOG Scheduler som er endel av IBM sitt ILOG CP\nomenclature{CP}{Begrensningsprogrammering (engelsk: Constraint programming)}. ILOG Scheduler er et C++ bibliotek som gjør det mulig å definere planleggingsbegrensninger i form av ressurser og aktiviteter. Planlegging er en prosess hvor det tildeles ressurser til aktiviteter og tid til de forskjellige aktiviteter slik at det ikke oppstår noen konflikt med begrensningene.\cite{Pape94implementationof} Automatisk planlegging er endel av det som kalles kunstig intelligens (AI) \nomenclature{AI}{Kunstig intelligens (engelsk: artificial inteligence)}.

Problemer i kategorien \textit{beregningsvitenskaps kompleksitets teori}\cite{compcomplextheory} og som også er minst like vanskelige å løse som de vanskeligste probleme i NP-kategorien, kalles \textit{ikke-deterministisk polynomtid problemer vansklige} (NP-hard) \nomenclature{NP-hard}{Ikke-deterministisk polynomtid hard (engelsk: non-deterministic polynomial-time hard)}. Det er forskjellige kategorier av NP-harde problemer, avgjøreksesproblemer, søkeproblemer og optimaliseringsproblemer. \cite{nphardwikipedia} Problemet i dette prosjektet er i kategorien optimaliseringsproblem. Fra wikipedia \cite{optimizationproblemwiki} er et optimaliseringsproblem: ''Et problem med å finne den beste løsningen ut ifra alle gyldige løsninger".

\subsubsection{Relatert arbeid}
Nuijten og Pape har brukt ILOG Scheduler til å løse \"Job Shop Scheduling\" problemet \nomenclature{JSSP}{Job Shop Scheduling Problem} og det er et problem som er NP-hard. Når problemer blir løst med Scheduler blir alle starttidene til aktivitetene, i en mulig løsning, som ikke er bevist om er en gyldig løsning tatt vare på i søkefasen. Begrensingsbasert planlegging blir ofte brukt for å redusere beregningstiden som trengs. Begrensingene reduserer domenet av variable. I industrien er ofte planleggingsproblemmer uført i dynamiske miljøer, som vil øke behovet for reaktive planleggingsalgoritmer. En mulighet for å bruke begrensningsprogrammering på en reaktiv måte er å overføre handlinger og deler av planleggingen som den er nå til begrensinger og domener. \cite{Nuijten:1998:CJS:594934.594971}

Taillard har sett på benchmarking i enkle planleggingsproblemer og sett på hvordan best mulig benchmarke vanskelige problemer. De har definert vanskelige problemer med å se på makespan og hvor langt unna den er den nedre grensen. De har tatt utgangspunkt i tre kjente begrensningsproblemer og laget benchmarksett til disse \cite{Taillard1993278}. Wallace har også sett på benchmarking og tatt for seg porsitive og negative sider ved benchmarking. Benchmarking har to hovedoppgaver, hvor det første er å dekke et representativt problem og bredt programmerings begrep. Den andre oppgaven er benchmarking av applikasjoner med \"unit tester\". Utfordringen med begge er å opprette en klar målsetting for ytelsen. Teoretisk sett er den mest ideele måten å benchmarke en applikasjon å implementere en løsning for hvert system og velge den beste, men dette er ikke alltid mulig i praksis. Måten da å gjøre det på i praksis er å benchmarke systemer med forhåndsdefinerte problemer og håpe resultatet kan bli overført til applikasjonen som krever det. Wallace tar også opp vanskeligheten med tidsmekanismer i høynivåspråk, da dette ofte er basert på CPU tid \cite{Wallace:2004:BCL:956860.956861}.
\begin{comment}
\begin{itemize}
\item (Laborie) IBM ILOG CP Optimizer for Detailed Scheduling Illustrated in Three problems \cite{Laborie:2009:IIC:1560579.1560593}
\begin{itemize}
\item ...
\end{itemize}

\item (Biskup) Enslig-maskin planlegging med læringsbetraktninger %\cite{Biskup1999173}
\begin{itemize}
\item \dots
\end{itemize}

\end{itemize}
\end{comment}