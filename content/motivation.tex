Prosjektoppaven er motivert av generell erfaring fra planleggingsprosesser i bedrifter og samfunnet forøvrig. I hverdagen har hver og en av oss mange planleggingsproblemer, alt fra buss- og togtabeller til personlige gjøremål som skal inn i en stram tidsplan. Nærligslivet  bedriver i varierende grad bemannings- og ressursplanlegging. Det å finne optimale løsninger på planleggingsoppgaver er viktig og ressursbesparende. I denne prosjektoppgaven er det planlegging i oljeindustrien som er i fokus.

Planlegging i oljeindustrien er viktig, da en på mest mulig effektiv måte må benytte seg av de ressursene som er tilgjengelige til enhver tid. Riktig planlegging hvor en tar hensyn til de begrensninger og ytre rammer en er stilt ovenfor, er en viktig faktor for å få utført oppgavene optimalt. Det er også mye penger involvert i oljeindustrien. Det å utføre vedlikeholdsaktiviteter på en ineffektiv måte, kan for selskapene både bli svært kostbart samt medføre stor risiko dersom sikkerheten ikke er godt nok ivaretatt. Det er derfor svært viktig å ha gode løsninger for planlegging av vedlikeholdsaktiviteter og ressurser. Målet til operatører er bl.a å minimere antallet farlige situasjoner og minimere fare for miljøskader samtidig som de ønsker å maksimere produksjon og dermed økonomisk resultat. På veien for å nå disse kompliserte og utfordrende målene, kan det i planleggingen oppstå konflikter i enkelte situasjoner (overfylte lokasjoner, manglende utstyrskrav, feil i rekkefølgen aktiviteter utføres, etc.).

Spesifikke planleggingssituasjoner i oljeindustrien har forskjellige forutsetninger i forhold til om det er offshore eller onshore. Offshore blir mange aktiviteter utført på et relativt lite lukket område. Onshore blir mange aktiviteter utføres på et relativt stort område. Selv om forutsetningene for disse er ganske forskjellige, så har de flere likhetstrekk:
\begin{itemize}
\item størrelse - antall aktiviteter kan være noen hundre til titusener, som gjør planlegging uoversiktelig.
\item kompleksitet - et stort antall begrensninger skal tilfredsstilles og gjør planlegging vanskelig.
\item dynamikk - avhengigheter som vær, logistikk og defekt utstyr endrer planleggingsforutsetningene.
\end{itemize}

\subsection{Målet med prosjektoppgaven}
Målet med prosjektoppgaven er å utvide den eksisterende Scheduler løsningen med en varmebegrensning og evaluere resultatene med og uten varmebegrensning. Mål:
\begin{itemize}
\item legge til \textit{varmebegrensning}
\item implementere i Scheduler
\item minimere \textit{makespan}
\end{itemize}

I den opprinnelige problemstillingen \cite{tvedtbezem} er noen aktiviteter relativt lite begrenset. Dette gjør at løsningsrommet er stort, og traverseringen opp og ned i søketreet tar lang tid. På tross av et antatt stort løsningsrom, så sliter den Scheduler implementerte løsningsstrategien med å finne løsninger i mange av probleminstansene. Denne prosjektoppgaven vil søke å finne ut:

\begin{itemize}
\item Vil flere begrensninger bidra til å finne flere løsninger?
\item Er det noe spesielt med akkurat disse probleminstansene eller er det implementasjon i Scheduler som er årsaken til at den Scheduler implementerte løsningenstrategien sliter med å finne løsninger i mange av probleminstansene?
\end{itemize}

Denne prosjektoppgaven har som hensikt å undersøke om løsningene blir bedre når et eksistrende planleggingsproblem får lagt til en ekstra begrensning, her kalt varmebegrensning. Ved å evaluere løsningene med og uten varmebegrensing, vil effekten av tillagt varmebegrensning fremgå. I tillegg fremgår effekten av lokasjon ved å sammenstille resultatene i denne oppgaven med resultatene i \cite{tvedtbezem}.