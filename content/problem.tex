Problemstillingen i denne oppgaven tar utgangspunkt i den opprinnelige problemstillingen til \bht. Beskrivelsen i det etterfølgende og frem til det fremkommer i etterfølgende tekst, er en oversettelse av \cite{tvedtbezem} til norsk. Problemet er på en fiktiv oljeplattform inndelt i et sett av lokasjoner. Utstyr som er krevd for vedlikehold er tilfeldig plassert rundt på plattformen og ulike aktiviteter skal planlegges. Aktivitetene blir opprettet med et gitt sett av ressurskrav og mulige avhengigheter til andre aktiviteter. Alle aktiviteter krever et mannskap til å utføre dem og en lokasjon til å bli utført på. I tillegg krever noen aktiviteter kranressurser, fordi tung løfting er involvert. Mannskaps- og kranressurser er knappe, det vil si de har begrenset tilførsel.

Så langt er problemet klassifisert som et \textit{''Resource-Constrained Project Scheduling Problem (RCPSP)"}\cite{RcpspPaperSchutt}, som kjennetegnes ved:
\begin{itemize}
\item Et sett av ressurser med en gitt kapasitet
\item Et sett av ikke-forstyrrede aktiviteter som er gitt en prosseseringstid
\item Et nettverk av begrensinger mellom aktiviteter
\item En mengde av ressurser som er krevd av aktivitetene
\end{itemize}

Det er en mengde planleggingsproblemer som ikke kan klassifiseres under beskrivelsen av RCPSP, selv om det også er et bredt antall planleggingsproblemer som gjør det. Det er mange tilleggsbegrensninger, typisk i oljeindustrien og andre store industrier, som ikke passer inn i denne klassifiseringen. Siden målet er å generere probleminstanser med begrensninger som finnes i industrien, så må det legges til andre mer komplekse begrensinger. Et eksempel  er sikkerhetsbegrensing rundt farlig arbeid, for eksempel kranbruk. I planleggingsløsninger i dag blir informasjon som sikkerhetsbegrensninger lagt til manuelt av de som planlegger aktivitetene på platformen. Ved å definere forutsetninger som aktiverer sikkerhetsbegrensninger, blir resultatet en veldefinert problembeskrivelse. En løsning til et problem $ S(P_{i}) $ er en planlegging hvor aktiviteter er tilegnet en starttid og begrensningene er gjeldende.

\subsection{Notasjoner og terminologi}
En probleminstans $ P $ inneholder aktiviteter som skal gjennomføres, ressurser som er påkrevd for å gjennomføre aktivitene og begrensninger som blant annet er begrensinger mellom aktiviteter og ressursbruk. Det blir skilt mellom forskjellige typer variable som \textit{avgjørelses-variable}\nomenclature{Avgjørings-variable}{På engelsk: decision variable}, \textit{konstanter} og \textit{avledet-variable}\nomenclature{Avlednings-variable}{Derived variable}. Et eksempel på en avgjørings-variabel er starttiden til en aktivitet $ Act_{i} $ betegnet som $ v_{sta}(Act_{i}) $. En aktivitets varighet blir betegnet som fast og er derfor en konstant, betegnet som $ c_{dur}(Act_{i}) $. Til slutt er det avledet-variable som for eksempel er en aktivitets sluttid, som er summen av starttiden og varigheten, som er betegnet $ w_{end}(Act_{i}) $. Objekter som aktiviteter og ressurser er skrevet med stor bokstav.

\subsection{Ressurser}
En \textit{lokasjon} $ Loc_{l} \in Locs = \{ Loc_{1},\dots,Loc_{n} \} $ er stedet hvor aktiviteter blir utført. Lokasjoner blir vist som ressurser, og varmebegrensning er en begrensning som viser maksimal kapasitet av hva lokasjonen tåler. Det er også begrensninger når farlig arbeid som tung løfting blir utført, og da er lokasjonen utilgjengelig for alle andre aktiviteter. Når en lokasjon er stengt på grunn av kranbruk, sier vi at en sikkerhetsone har blitt opprettet.

\textit{Mannskaper} er ansvarlige for utførelse av aktiviteter. Et mannskap er betegnet $ Crew_{j} \in Crews = \{ Crew_{1},\dots,Crew_{n} \} $. Mannskaper er vist som ressurser, og har fått varmebegrensning. Varmen blir brukt til å begrense hvor mange mannskaper som kan være på en lokasjon på samme tid.

En \textit{kran} $ Crane_{k} \in Cranes = \{ Crane_{1},\dots,Crane_{n} \} $ er en potensiell ressurs for aktiviteter. Noen aktiviteter trenger kran og alle probleminstanser har et mindre antall av aktiviteter som krever kranbruk. Kraner er eneressurs, \nomenclature{Eneressurs}{På engelsk: Unary resource} som betyr at de kun kan utføre en aktivitet av gangen. En aktivitet som krever kran, spesifiserer ikke en spesifikk kran, men sier kun den trenger kran. En gyldig løsning må derfor tildele en kran til alle aktiviteter som krever kran fra et sett av kraner tilgjengelig, gitt av $ v_{crane}(Act_{i}) \in Cranes $. Dette gjør settet av kraner til en alternativ ressurs.

Kraner har en lokasjon $ c_{loc}(Crane_{k}) \in Locs $, og hver lokasjon kan bare ha en kran. På grunn av at tung løfting er farlig arbeid, er kranbruk omgitt med sikkerhetssoner. Disse sikkerhetssonene er satt til både lokasjonen hvor aktiviteten som krever kranbruk er utført og kranens egen lokasjon. Sikkerhetssonen som blir satt vil derfor variere ut ifra hvilken kran som er tilegnet til aktiviteten.

\subsection{Aktiviteter}
En \textit{aktivitet} $ Act_{i} \in Acts = \{ Act_{i},\dots,Act_{n} \} $ kommer med en startvariabel, en konstant varighet og ressurskrav. Initielt er domenet til startvariabelen $ v_{sta}(Act_{i}) \in [ 0, c_{hor}(P)) $, hvor horisonten som indikerer planleggingens maksimale fullføringstid er gitt ved $ c_{hor}(P) = \sum_{i} c_{dur}(Act_{i}) $.

En aktivitet $ Act_{i} $ krever et mannskap $ c_{crew}(Act_{i}) \in Crews $ for å utføre aktiviteten og en lokasjon $ c_{loc}(Act_{i}) \in Locs $ til å bli utført på. En aktivitet er avhenger av hvert enkelt medlem av et mannskap og det er ikke mulig å samle ressurser for å redusere varigheten. Kraner er den siste ressursen som er tilgjengelig, men er ikke nødvendig for alle aktivitetene.

I tillegg til ressurskravene, kan en aktivitet være avhengig av andre aktiviteter, det betyr at en aktivitet ikke kan starte før en annen aktivitet er ferdig utført.

\subsection{Begrensinger}
\textit{Avhengigheter} mellom aktiviteter er vanlig i industrien. En vedlikeholdsaktivitet kan for eksempel være avhengig av både levering av reservedeler og stillasbygging for å sikre tilgang til området hvor vedlikeholdet skal gjøres. Forholdet som viser at aktivitet $ Act_{i'} $ avhenger av aktivitet $ Act_{i} $ er uttrykt ved følgende begrensning: 
\begin{equation}
w_{end}(Act_{i}) \leq v_{sta}(Act_{i'})
\end{equation}

En \textit{kummulativ ressursbegrensing} påføres alle mannskaper for å være sikkert på at den totale ressursbruken ikke overstiger tilgjengelig kapasitet. Det er utrykt ved: 
\begin{equation}
\forall t,j : \sharp \{ Act_{i} | t \in [(v_{sta}(Act_{i}), w_{end}(Act_{i})) \wedge c_{crew}(Act_{i}) = Crew_{j}] \} \leq c_{cap}(Crew_{j})
\end{equation}
hvor $ c_{cap}(Crew_{j}) $ er kapasiteten av j's mannskap.

Kraner er unikt induvidielle og er derfor modellert som et sett av eneressurser. Begrensingene tar for seg hvis to aktiviteter er tilegnet den samme kranen, så kan de ikke bli utført samtidig. Vi starter ved å definere den underliggende overlappingen uttrykt som to aktiviteter som overlapper i tid: 
\begin{equation}
overlap(Act_{i},Act_{i'}) \equiv \exists t : v_{sta}(Act_{i}),v_{sta}(Act_{i'}) \leq t < w_{end}(Act_{i}),w_{end}(Act_{i'})
\end{equation}
Den gjensidge uttelukkelsen opprettet av eneressursbegrensningen blir da: 
\begin{equation}
\forall i,i' \neq i : c_{crane}(Act_{i}) = v_{crane}(Act_{i'}) \rightarrow \neg overlap(Act_{i},Act_{i'})
\end{equation}
for alle aktiviteter som krever kran.

\textit{Sikkerhetsbegrensningene} er uttrykt i form av lokasjonen til aktiviten som krever kran og lokasjonen til den valgte kranen. Den første lokasjonen er kjent på forhånd, mens den andre avhenger av hvilken kran som blir brukt. Tilfellet at begrensingene i problemet endrer seg etter hvert som avgjørelser tas, er interessant på grunn av den tillagte kompleksiteten det medfører.

Sikkerhetsbegrensningene utelukker bruken av lokasjonen hvor en aktivitet som krever kran befinner seg:
\begin{equation}
\forall i,i' \neq i : c_{crane}(Act_{i}) \wedge c_{loc}(Act_{i}) = c_{loc}(Act_{i'}) \wedge \neg overlap(Act_{i},Act_{i'})
\end{equation}
Når sikkerhetsbegrensningene utelukker bruken av lokasjonen til denne kranen er gitt ved:
\begin{equation}
\forall i,i' \neq i : v_{crane}(Act_{i}) = Crane_{j} \wedge c_{loc}(Act_{i'}) = c_{loc}(Crane_{j}) \rightarrow \neg overlap(Act_{i},Act_{i'})
\end{equation}

Så langt er formler og tekst hentet fra \cite{tvedtbezem}. I det neste tillegges en ny type begrensning kalt \textit{varmebegrensning}. Med dette, så menes ikke varme i tradisjonell forstand, men som en måte å kunne sette en verdi på et mannskap og en kapasitet på en lokasjon. Det kan være lokasjoner som av forskjellige årsaker (begrenset plass, restriksjoner til lokasjonen, etc.) ikke kan ha ubegrenset med mannskap til å jobbe der samtidig. Forskjellige mannskaper kan også ha forskjellig kapasitet ut ifra hva slags arbeid de utfører. Mannskaper som driver arbeid som sveising, kan for eksempel ha en høyere varmeverdi enn et mannskap med elektrikere. Grunnen til at sveisere kan ha en høyere varmeverdi er fordi det er en aktivitet som gjør det ugunstig å ha for mange på lokasjonen. Det kan være grunner til farlige gasser, eksplosjonsfare osv. De er uttrykt som kummulativ ressursbegrensning og er påført lokasjon for å være sikkert på at total varmebruk ikke overstiger varmekapasiteten tilgjengelig på hver lokasjon. Varmebegrensingen skal også etterhvert kunne erstatte sikkerhetsbegrensningene på for eksempel kranbruk, ved at lokasjoner med kran får en varmekapasitet og aktiviter som bruker kran bruker opp denne varmekapasiten på lokasjonen. Den er uttrykt ved:
\begin{equation}
\begin{split}
\forall t,l: \sum\{c_{heat}(Crew_j) \mid t \in [ v_{sta}(Act_{i}), w_{end}(Act_{i})) \wedge c_{crew}(Act_{i}) = Crew_{j} \\
\wedge c_{loc}(Act_{i}) = Loc_{l} ] \} \le c_{heatcap}(Loc_{l})
\end{split}
\end{equation}

\subsection{Målfunksjon}
Målet er å minimalisere makespan $ w_{ms}(P) $ eller varigheten av planleggingen, som er definert ved:
\begin{equation}
w_{ms}(P) = max_{i} \{ w_{end}(A_{i}) \} \in [0,c_{hor}(P)]
\end{equation}
Dette uttrykker at makespanet er likt den siste slutten eller fullføringstiden i settet av aktiviteter.

\subsection{Probleminstanser}
Problemene er beskrevet ved størrelsen fastsatt av det totale nummeret av aktiviteter, $ \#Acts \in \{ 50,60, 70, 80, 90, 100, 200, 300, 400, 500, 600, 700, 800, 900, 1000, 5000 \} $ og kraner, $ \#Cranes = [2,3] $. Det ble generert totalt 5 probleminstanser for hver av de 32 problem størrelsene, som summert opp blir 160 instanser.

Probleminstansene ble tilfeldig generert, ved å tildele mannskaper til aktiviteter, lokasjoner til aktiviteter, lokasjoner til kraner, avhengigheter mellom aktiviteter og aktiviteter som trenger kran. Når instansene ble generert, er det spesifisert at det ikke skal forekomme sirkulasjoner på aktivitetsavhengigheter og at det ikke skal være mer enn en kran på en lokasjon. Så alle 160 instansene er gyldige.

Alle problemene har 10 lokasjoner, som er redusert fra 25 i de opprinnelige probleminstansene til \bht. Med aktiviteter fordelt utover 25 lokasjoner, så ville det vært så få aktiviteter på hver lokasjon at varmebegrensingen til en lokasjon aldri ville blitt oversteget. Det er 4 forskjellige mannskaper med kapasitet $ c_{cap}(Crew_{j}) \in [2,3] $ tatt fra en uniform fordeling. Domenet for aktivitenes startvariabel er generelt $ v_{sta}(Act_{i}) = [0,c_{hor}(P)] $ og de konstante varighetene $ c_{dur}(Act_{i}) $ er tilfeldig tatt fra en uniform fordeling i området $ [1,6] $ tidssteg. Omtrent 20\% av aktivitetene er tilfeldig valgt til å bruke kran og omtrent 10\% av aktivitetene er begrenset ved avhengighet til en annen aktivitet. I det første settet med probleminstanser ble mannskapers varme tilfeldig generert i området: $c_{heat}(Crew_{j}) \in [5,15]$ og lokasjoners varmekapasitet ble tilfeldig generert i området: $c_{heatcap}(Loc_{l} \in [10,20]$. Det ble generert et nytt sett med probleminstanser, hvor mannskapers varme er tilfeldig generert i området: $c_{heat}(Crew_{j}) \in [1,5]$, mens lokasjoners varmekapasitet er tilfeldig generert i området: $c_{heatcap}(Loc_{l}) \in [6,10]$.